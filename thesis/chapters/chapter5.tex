
%But first I need a simple statement to remind me that I am making progress and that my work is going towards a good end. It makes sense in the context of the program. The more I delved into research the more ambitious my project became. In the course of my graduate career I became a different person in many way, more like I shed the old person in order to survive. I realize that I need a career or job or at least an activity (while basic needs for food shelter and love met elsewhere) that I am grateful for, and I think that means it needs to be challenging and supportive of my growth in the process. I am still a biologist and I think that means I am a student of what works. I think the field of systems biology best represents my view of biology and many fields of inquiry within it are opening due to advances in computing power (size and speed) that enable previously advanced or theoretical mathematics to come to bear on very old problems. Even setting -omics aside, the current state of microscopy and automated image analysis allow for information capture that far exceeds currently utilized analysis methods. I feel like nonparametric/nonlinear/ new machine learning methods are the answer. Mathematics is the new microscope. I need to learn tools and techniques and this masters project was a chance to practice and demonstrate them.

\subsection{Cheng's results compared to mine}
It is highly desirable to develop computational models that can predict the inhibitive effect of a compound against a specific CYP isoform. In this Cheng's study inhibitor predicting models were developed for five major CYP isoforms, namely 1A2, 2C9, 2C19, 2D6, and 3A4, using a combined classifier algorithm on a large data set containing more than 24,700 unique compounds, extracted from PubChem. The combined classifiers algorithm is an ensemble of different independent machine learning classifiers including support vector machine, C 4.5 decision tree, $/kappa$-nearest neighbor, and naive Bayes, joined by a back-propagation artificial neural network (BP-ANN). Those models were validated by 5-fold cross-validation with a diverse validation set composed of about 9000 diverse unique compounds. The range of the AUROC for the validation set was 0.764 to 0.886. The overall performance of combined classifiers fused by BP-ANN was superior to that of three classic fusion techniques (Mean, Maximum, and Multiply). Cheng et al. claim these classification models are applicable for virtual screening of the five major CYP isoforms inhibitors or can be used as simple filters of potential chemicals in drug discovery.\cite{Cheng2011}