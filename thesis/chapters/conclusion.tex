

QSAR models and their value in informing regulatory decision making will likely increase with the standardization of analytical approaches, more complete and reliable data collection methods and a better understanding of toxicity mechanisms in the role of disease as well as individual susceptibility to adverse clinical events. \cite{Kruhlak2012}

I have demonstrated several methods for building models for binary classification for QSAR of cytochrome P450 isozymes. They all do pretty well and are significantly better than randomly guessing.
There are many more machine learning methods to try, of particular interest are neural networks and more Bayesian methods. The data and code for model comparison for this study is available online, so there are no artificial barriers to benchmarking further results.


