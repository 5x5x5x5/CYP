

QSAR models and their value in informing regulatory decision making will likely increase with the standardization of analytical approaches, more complete and reliable data collection methods and a better understanding of toxicity mechanisms in the role of disease as well as individual susceptibility to adverse clinical events. \cite{Kruhlak2012}

I have demonstrated several methods for building models for binary classification for QSAR of cytochrome P450 isozymes. They all do pretty well and are significantly better than randomly guessing.
There are many more machine learning methods to try, of particular interest are neural networks and more Bayesian methods. The data and code for model comparison for this study is available online, so there are no artificial barriers to benchmarking further results.


%\subsection*{Not Related - Systems Biology}

%The ultimate goal of systems biology is an understanding of physiology and disease across the multiple hierarchical levels of organization, from chemical and molecular interactions to pathways and pathway networks, at the cell-cell and issue level, organs and organ systems and, ultimately, to the functioning of the whole organism.(Ellen Berg 2014)

%Systems biology research encompasses the generation of high­throughput datasets of system components (omics data), experimental methods of analysis and data integration, as well as the development and application of network approaches and computationally derived models(Berg 2014)

%In pharmaceutical research, systems biology efforts are directed towards the identification of drug targets, the development of novel therapeutics and new indications for existing drugs.

%Studies tend to be compound-centric, concerned with the identification and characterization of small molecules or biologics that selectively inhibit (or activate) specific molecular targets or pathway mechanisms. Thus, studies related t drug mechanisms of action and those that support drug development goals, such as clinical indication selection and patient stratification, are of particular interest. (Berg 2014)

%Omics tools, developed over the past several decades, can provide global information on the levels and dynamic changes in cellular and tissue components at specific time points insamples from cell­based assays, precinical animal models or human studies. Omics data sets derived from transcriptomics, proteomics, and metabolomics are being used and integrated with each other as well as genomics information and other data types to construct models of cell signaling, pathway and disease networks to identify new targets as well as to help better understand and predict drug action in vivo.
% ""In addition to experimentally derived data sets, there is a wealth of literature information and accumulated knowledge that can be incorporated by converting to some type of formal representation. This is accomplished through the use of a defined ontology by expert curation and/or natural language processing(NLP) - based methods into a series of semantic statements.""(Berg2014)
