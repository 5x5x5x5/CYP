This project compares methods of Cytochrome P450 inhibitor prediction based on compound structures. CYPs are a natural first choice in which to  develop \textit{in silico} models because of their central role in drug candidate rejection due to adverse drug-drug interactions. Several non-linear, high-dimensional classification models were built and compared using a large, publicly available high throuhput screening luminescence assay (PubChem AID1851) against five CYP isozymes (1A2, 2C9, 2C19, 2D6 and 3A4). The methods compared are a Bayseian binary QSAR method from the Molecular Operating Environment, and 3 standard machine learning methods implemented in the Python programming language; $\kappa$-Nearest Neighbor, Random Forests, and Support Vector Machines. They all performed well, with the methods implemented in freely available software performing as well or better than the one in software that is widely accepted in industry.