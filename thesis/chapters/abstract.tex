This project compares Cytochrome P450 inhibitor prediction methods based on compound structures. CYPs are a natural first choice in which to  develop \textit{in silico} models because of their central role in drug candidate rejection due to adverse drug-drug interactions. Several non-linear, high-dimensional classification models were built and compared using a large, publicly available high throuhput screening luminescence assay (PubChem AID1851) against five CYP isozymes (1A2, 2C9, 2C19, 2D6 and 3A4). The methods compared are a bayseian binary QSAR method from the Molecular Operating Environment, and 3 standard machine learning methods implemented in the Python programming language; $\kappa$-Nearest Neighbor, Random Forests, and Support Vector Machines. They all did better than random guessing.